%% start of file `template.tex'.
%% Copyright 2006-2010 Xavier Danaux (xdanaux@gmail.com).
%% Copyright 2010-2011 Mark Liu (markwayneliu@gmail.com).
%
% This work may be distributed and/or modified under the
% conditions of the LaTeX Project Public License version 1.3c,
% available at http://www.latex-project.org/lppl/.

\documentclass[11pt,a4paper,sans]{moderncv}

\usepackage{verbatim}

% moderncv themes
\moderncvstyle{classic}
\moderncvcolor{blue}

% character encoding
\usepackage[utf8]{inputenc}                   % replace by the encoding you are using

% adjust the page margins
\usepackage[scale=0.8]{geometry}
%\setlength{\hintscolumnwidth}{3cm}						% if you want to change the width of the column with the dates
%\AtBeginDocument{\setlength{\maketitlenamewidth}{6cm}}  % only for the classic theme, if you want to change the width of your name placeholder (to leave more space for your address details
%\AtBeginDocument{\recomputelengths}                     % required when changes are made to page layout lengths


% personal data
\firstname{RECursion}
\familyname{}
%\address{108,J.C Bose Hall of Residence,NIT DURGAPUR,WB,713209
%}{}     optional, remove the line if not wanted
%\mobile{\textbf{+91-9143227223/+91-9051035048}}                    % optional, remove the line if not wanted
%\email{satraprathore@gmail.com}                      % optional, remove the line if not wanted
%\homepage{http://markliu.me}                % optional, remove the line if not wanted
%\extrainfo{\url{http://markliu.me}} % optional, remove the line if not wanted

% to show numerical labels in the bibliography; only useful if you make citations in your resume
%\makeatletter
%\renewcommand*{\bibliographyitemlabel}{\@biblabel{\arabic{enumiv}}}
%\makeatother

%\nopagenumbers{}                             % uncomment to suppress automatic page numbering for CVs longer than one page
%----------------------------------------------------------------------------------
%            content
%----------------------------------------------------------------------------------
\begin{document}
\maketitle

\section{Competitive programming}
\cventry{}{Competitive programming is solving well-defined problems by writing computer programs under specified limits.}{It involves the host presenting a set of logical or mathematical problems to the contestants (who can vary in number from tens to several thousands), and contestants are required to write computer programs capable of solving each problem. Judging is based mostly upon number of problems solved and time spent for writing successful solutions, but may also include other factors (quality of output produced, execution time, program size, etc.)}{}{}{}


%\cvline{advisor:}{\small Professor Eric Bach}
%\cventry{2003--2007}{BS, Computer Science and Engineering}{UCLA}{Los Angeles, CA}{}{}   arguments 3 to 6 can be left empty

%\cvline{honors:}{\small Cum Laude}


\section{ WHY Competitive programming}



%\cventry{2012--Present}{ (\url{http://pacescheduler.com/})}


\cventry{}{•	You get better applying theoretical concepts such as data structures, computational complexities, dynamic programming, graph theory, etc. which can be very useful when writing performance-critical software.}{}{}{}{}
\cventry{}{•	Competitive programming communities have leaderboards that list competitors’ performance in the contests. A good ranking demonstrates that you can solve problems under pressure, and it shows where you stack up compared to your peers. }{}{}{}{}
\cventry{}{•	It's a proof that you have brainpower; any potential employer/investor/business partner would appreciate that skill.}{}{}{}{}
\cventry{}{•	You learn how to write code and do team work under time pressure. It's actually great for life in general.}{ }{}{}{}
\cventry{}{•	If you become a beast at it, you will get payed for it and do international travel.}{}{}{}{}
\cventry{}{•	The environment at these competitions is great, you get the opportunity to make a lot of smart friends. Some examples of people who used to be competitive programmers in their college years include Adam D'Angelo, Craig Silverstein and Nikolai Durov; among others.}{ }{}{}{}
\cventry{}{•	And most importantly getting an awesome job.}{}{}{}{}


\section{Role of RECursion}
%\subsection{ }
\cvline{ }{ RECursion – NIT Durgpur , codechef campus chapter is a small initiative taken by the students of NIT Durgapur with the aim of introducing competitive programming to the students in their freshman year. Inculcating the urge to become a good coder in the very first year helps them make their focus very clear and  developing a healthy coding culture in our college.
 In order to accomplish this, we conduct classes every week. Starting from the scratch we teach different algorithms and approach towards the problems. To keep their vigor and passion for coding intact contests are held every month and the good performers are awarded with goodies and cash prizes.The classes which are held twice a week had positive outcomes. Participation in coding competitions have increased incredibly. 23(Kharagpur regional) and 16(Amritapuri regional) teams registered for ACM-ICPC 2014 whereas earlier hardly 5 to 6 teams had registered.  }



\end{document}
